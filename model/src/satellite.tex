\section{Introduction}
\subsection{Contexte}
-Blabla : objectifs du projets ...
-Rappel des données et des outils
\subsection{Problématique}
-Blabla sur l'actualité
-En fonction des résultats ou bien 
- "Lien entre liberté civile et données démographique, santé, armée, économie, sociale..."?
- Ensuite chercher à trouver (générer) les indices de liberté à partir des résultats (liens, modèles) établis
\section{Détermination du champ d'étude}
\subsection{Identification des attributs pertinents}
-Visualisation sous forme de diagrammes (lignes parallèles, scatter...)
Blabla plus interessant
-Justification des choix d'attributs pour chaque axe de recherche (santé, démographie...)
\subsubsection{Social}
\begin{description}
\item [Adolescent fertility rate] 
Voir yoyo
\item [Worker's remittances and compensation of employees]
Je ne sais pas ce que c'est
\item [Internet users]
Les aspects de censure liés aux dictatures se traduiraient par un usage très limité d'internet. Aussi le nombre d'internautes 
au sein d'une dictature se réduisant aux membres du régime, on s'attend à trouver une relation proportionnelle entre l'indice de liberté et le nombre d'internautes. 
\item [Mobile cellular suscriptions]
Le fort contrôle des moyens de communication de la part d'un régime totalitaire pourrait induire une utilisation de la télphonie mobile très limitée. //reformuler
 
\end{description} 

\graph{correlation_social}{Corrélation des différents attributs choisis dans la catégorie \emph{Social}}
Les attributs ayant trait aux nouvelles technologies de l'information sont linéairement corrélés, pour des raisons évidentes. De manière plus surprenante, la fécondité des adolescents semble se rapprocher d'une fonction linéaire décroissante du taux d'accès à Internet\ldots
Sans nous livrer à des conclusions trop hâtives, nous ne considérerons cependant que le taux d'accès Internet en tant qu'attribut représentant les trois.

\subsubsection{Santé} 
\begin{description}
\item [Immunization]
Cet attribut est étroitement lié aux impacts de la liberté sur les aspects économiques d'un pays. Un pays totalitaire
accorderait peu d'importance aux achats (imports ?) de vaccins contrairement aux investissements de l'armement militaire.
\item [Life expectancy]
On pourrait croire que les libertés d'un pays influence l'esperance de vie de sa population. Cette hypothèse serait une conséquence
des autres attributs qui lieraient indice de liberté aux aspects économiques et sociales d'un pays. En d'autres termes, moins un pays est libre plus sa situation sociale et sanitaire se dégrade, plus l'espérance de vie décroit.
\item [Mortality rate]
Un pays dépourvu de liberté pourrait avoir un taux de mortalité important dans la mesure où les mouvements des opposants sont réprimés par des peines de mort.
Les engagements militaires d'un tel pays entraineraient des pertes humaines considérables et donc un taux de mortalité élevé.
\item [HIV]
Un indice de liberté très bas augementerait le nombre de personnes atteintes de maladies. En particulier le SIDA.
\end{description}
 
\subsubsection{Armée}
\begin{description}
\item [Adolescent fertility rate]
Cf plus haut.
\item [Military expenditure]
On prévoit de trouver un lien très fort entre les dépenses militaires et l'indice de liberté d'un pays. Un pays totalitaire a de très importants dépenses militaires et inversement.
\item [Fertility rate, total]
On s'attend à trouver un lien entre le taux de fécondité et l'indice de liberté. Un indice de liberté bas se traduirait peut être par un taux de fécondité bas également et inversement.
\item [Life expectancy]
Cf plus haut.
\item [Surface]
Un grand pays en terme de surface pourrait être difficilement controlable par un régime totalitaire. Il posséderait donc un indice de liberté plus important qu'un pays plus petit en surface.
\end{description}
 
 
\subsubsection{Démographie} 
\begin{description}
\item [Fertility rate]
Cf plus haut
\item Adolescent fertility
\item Population totale
\item Population growth
\item Life expectancy
\item Surface
 
\end{description}
 
\subsubsection{Economie}
\begin{description}
\item [Voir knime] 
\end{description}

\subsection{Introduction d'un nouvel attribut}
-Blabla site internet, organisation
-Pb : absence données sous forme de fichiers exploitable directement (cvs) d'ou le script (merci yoyo)
\subsection{Choix des pays à étudier}
-décompostion géographique ??

\subsection{Elimination des outliers}
-Diagrammes (Scatter Plot, Boites à moustache...)
-1D, 2D plus de 2D(cahier de parrain)
\subsection{Discrétisation de la dimension \og liberté \fg}
Blabla
-Clustering hiérarchique
-K-Means (ne pas oublier de parler de la stabilité)
\section{Classification non supervisée}
\section{Classification supervisée}
\section{Conclusion}



L'objectif de ce projet est d'exploiter un jeu de données	
	
-Objectifs

\section{Selection}

\section{Missing value}
\section{Correlation}
\section{Outliers}

\section{Normalisation}


Décomposer les pays en cluster en fonction des dépenses militaires


