\usepackage[french]{babel}
\usepackage[utf8]{inputenc}
\usepackage[T1]{fontenc}

\usepackage{verbatim}
\usepackage{graphicx}
\usepackage{subfig}
\usepackage{float}
\usepackage{xspace}

\usepackage{hyperref}

\usepackage[top=20mm, bottom=20mm, left=25mm, right=25mm]{geometry}


\usepackage{verbatim}
\usepackage[table]{xcolor}
\definecolor{bleugris}{rgb}{.2,.4,.5}

\definecolor{colKeys}{rgb}{0,0,1} 
\definecolor{colIdentifier}{rgb}{0,0,0} 
\definecolor{colComments}{rgb}{0,0.5,1} 
\definecolor{colString}{rgb}{0.6,0.1,0.1} 

\usepackage{listings}

% Permet l'ajout de code par insertion du fichier le contenant
% Les arguments sont :
% $1 : nom du fichier à inclure
% $2 : le type de langage (C++, C, Java ...)
\newcommand{\addCode}[2]{%
    \lstinputlisting[language = #2,%
    identifierstyle=\color{colIdentifier},%
    basicstyle=\ttfamily\scriptsize, %
    keywordstyle=\color{colKeys},%
    stringstyle=\color{colString},%
    commentstyle=\color{colComments},%
    columns = flexible,%
    %tabsize = 8,%
    showspaces = false,%
    numbers = left, numberstyle=\tiny,%
    frame = single,frameround=tttt,%
    breaklines = true, breakautoindent = true,%
    captionpos = b,%
    xrightmargin=10mm, xleftmargin = 15mm, framexleftmargin = 7mm]%
	{#1}
}

\newcommand{\graph}[2]{
	\begin{figure}[H]
		\centering
		\includegraphics[width=\textwidth]{#1}
		\caption{#2}
	\end{figure}
}

\graphicspath{{../pics/}}
